\documentclass[12pt, leterpaper, twoside]{article}
\usepackage[utf8]{inputenc}
%\usepackage[latin1]{inputenc}
\usepackage[spanish,english]{babel}
\usepackage{listings}
\usepackage{xcolor} % for setting colors

\usepackage{color}

\definecolor{dkgreen}{rgb}{0,0.6,0}
\definecolor{gray}{rgb}{0.5,0.5,0.5}
\definecolor{mauve}{rgb}{0.58,0,0.82}



% set the default code style
\lstset{
    frame=tb, % draw a frame at the top and bottom of the code block
    tabsize=4, % tab space width
    showstringspaces=false, % don't mark spaces in strings
    numbers=left, % display line numbers on the left
    commentstyle=\color{gray}, % comment color
    keywordstyle=\color{blue}, % keyword color
    stringstyle=\color{red} % string color
}

\title{Métodos númericos}
\author{Daniela Trespalacios A}
\date{8 de febrero de 2017}



\begin{document}
\begin{titlepage}
    \maketitle
\end{titlepage}

%\begin{lstlisting}[language=C++, caption={C++ code using listings}]
%#include <iostream>
%int main()
%{
 %   // print hello to the console
%    std::cout << "Hello, world!" << std::endl;
%    return 0;
%}
%end{lstlisting}

%\begin{lstlisting}[language=Java, caption={Java code using listings}]
%public class Hello
%{
%    public static void main(String[] args)
%    {
%        // print hello to the console
%        System.out.println("Hello, world!");
%    }
%}
%\end{lstlisting}
\lstinputlisting[language=c++]{punt_02_b_gauss_eliminacion.cpp}[caption=My Javascript Example]

%\lstinputlisting[keywordstyle=\color{blue},commentstyle=\color{dkgreen},stringstyle=\color{mauve},breakatwhitespace=false,breaklines=true,rulecolor=\color{black},frame=single,showtabs=false,showstringspaces=false,showspaces=false,numbersep=5pt,numberstyle=\tiny\color{gray},basicstyle=\footnotesize,captionpos=b,caption=Elemento mayor,language=c++]{01_elemento_mayor.cpp}


%\lstinputlisting[keywordstyle=\color{blue},commentstyle=\color{dkgreen},stringstyle=\color{mauve},breakatwhitespace=false,breaklines=true,rulecolor=\color{black},frame=single,showtabs=false,showstringspaces=false,showspaces=false,numbersep=5pt,numberstyle=\tiny\color{gray},basicstyle=\footnotesize,captionpos=b,caption=Gauss eliminación,language=fortran]{02_a_gauss_eliminacion.f90}


%\lstinputlisting[keywordstyle=\color{blue},commentstyle=\color{dkgreen},stringstyle=\color{mauve},breakatwhitespace=false,breaklines=true,rulecolor=\color{black},frame=single,showtabs=false,showstringspaces=false,showspaces=false,numbersep=5pt,numberstyle=\tiny\color{gray},basicstyle=\footnotesize,captionpos=b,caption=Inverse of A by Gauss-Jordan elimination,language=c++]{02_b_inverse_by_Gauss-Jordan_elimination.cpp}




\end{document}
